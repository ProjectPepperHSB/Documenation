\abstract{
    Diese Dokumentation ist im Rahmen des Bachelorprojektes von
    Benjamin T. Schwertfeger, Jacob Menge und Kristian Kellermann entstanden
    und dient neben der Dokumentation des selbst gewählten Projektes ebenfalls als
    Anforderung für das Bestehen des Bachelorstudiums.\\

    Die Studenten haben sich dazu entschieden, eine Anwendung für den humanoiden Roboter Pepper,
    welcher sich derzeit im Besitz der Hochschule Bremerhaven befindet zu entwickeln. Hierzu
    werden verscheidene Anwendungsfälle implementiert, welche durch eine selbst implementierte
    Web-Schnittstelle in Form einer Backend Webanwendung unterstützt wird. \\

    Ziel ist es, dem Roboter das Interagieren mit Menschen beizubringen, sodass sinnvolle
    Interaktionen zwischen Mensch und Maschine zustande kommen. Zudem sollen mit Hilfe der
    Webanwendung zusätzlich dynamisch Informationen an Pepper wetergegeben und von ihm abgerufen
    werden, sodass Informationen zu Interaktionen, wie Dauer, Intensität und Erfolg des
    Gespräches gespeichert und ausgewertet werden können.\\

    Dieses Projekt wird von Frau Prof. Dr. Nadja Petram begleitet.
    Es wurdne keine Rahmenbedingungen festgelegt.
}