\newcommand{\scripts}{Kapitel 6.}

\chapter{Skripte und Erweiterungen}
\label{chapter:scripts}
\lhead{\scripts \emph{Skripte und Erweiterungen}}

\section{Allgemein}
Neben der Anwendung für den Roboter Pepper und der Webanwendung haben wir weitere Skripte,
hauptsächlich in Python, geschrieben. Diese sind in einem separaten Repository unter
\href{https://github.com/ProjectPepperHSB/Backend-Services}{https://github.com/ProjectPepperHSB/Backend-Services}
zu finden. In diesem Repositry befindet sich, wie in jedem anderen unserer Projekte auch, eine README.md, welche
einen Einstieg in die Installation und Anwendung der einzelnen Skripte ermöglicht.

Es sind mehrere Verzeichnisse angelegt, welche separate Schwerpunkte beinhalten, auf welche wir in den folgenden
Abschnitten genauer eingehen wedern. Auch diese bieten eigene README.md Dateien, welche die Installation, sowie den Zweck aufzeigen.

In jedem dieser kleineren Module befindet sich eine \verb|install.sh|, sowie eine \verb|reguirements.txt| Datei, welche
zusammen ausgeführt werden, um die für das Skript benötigten Packages zu installieren, sofern diese noch nicht vorhanden sind.

Es befindet sich auch ein Verzeichnis mit Namen ``analysis'' in diesem Repository, jedoch gehen wir
darauf im Kapiel \ref{chapter:big-data} genauer ein.

\section{Skripte zur Generierung von Dummy Konversationen}
\label{sec:dummy-data}
Aufgrund der andauernden Einschränkungen der Regierung, ist es uns nicht möglich, Pepper an der Hochschule im vollem Umfang
auszutesten. Somit können wir Pepper und seine Interaktionen nicht an einer großen Menge an Studierenden und Interessierten austesten.
Da wir jedoch den Schwerpunkt Big Data mit in unserem Projekt einfließen lassen wollen und unsere Webanwendung, welche wir im Kapitel \ref{chapter:webapp}
besprochen haben genau für das Sammeln und zur Verfügung stellen von Daten ausgelegt haben, war es für uns ganz klar, dass
wir uns selbst Daten generieren müsssen.

Hierfür wurde ein Skript mit dem Namen \verb|create-dummy-data.py| geschrieben, welches über die Kommandozeile ausgeführt werden kann
und mehrere Parameter übergeben bekommt.

\begin{lstlisting}
    ~$ python3 create-dummy-data.py -n 1000 --prod
\end{lstlisting}

Das Flag \verb|n| gibt die Anzahl der zu generierenden Konversationen an. Das zweite Flag \verb|--prod| ist optioanl und
sorgt dafür, dass die Daten, welche innerhalb des Skriptes generiert werden, an die URL der
laufenden Webanwendung auf dem Hochschulserver gesendet werden. Ohne diesen Flag, werden die Datenreihen an den Localhost geschickt.
Sollte dies nicht einwandfrei laufen, so ist die Webanwendung höchstwahrscheinlich nicht aktiv.

Da die \verb|requests| Library in Python nur synchrone Prozesse unterstützt und dies bei 1000 Anfragen etwas Zeit in Anspruch nimmt,
haben wir dies mit der Library Joblib parallelisiert, sodass 6 Prozesse gleichzeitg die Generierung der Daten, sowie das Übermitteln
an die Webanwendung übernehmen. Aufgurnd der Beschränkungen des Hochschulservers Hopper ist es nicht möglich, noch mehr
gleichzeitige Anfragen zu schicken. Dies ist eine Sicherheitsmaßnahme zur Abwendung von DOS Attacken.


\section{Skripte für die Bereitstellung des Mensaplans}

\section{Skripte für die Bereitstellung des Routenfinders}

\section{Sonstiges}
% timetable data