\newcommand{\chaptermotivation}{Kapitel 2. }

\chapter{Motivation und Zielsetzung}
\label{sec:Motivation-und-Zielsetzung}
\lhead{\chaptermotivation \emph{Motivation und Zielsetzung}}

Wie alle Bachelorstudierende kommt irgendwann die Zeit, in der es darum geht, ein Projekt zu finden, mit welchem man sich über einen längeren Zeitraum beschäftigt, um dies im Rahmen des Studiums schriftlich niederzulegen. Wir drei sind von der Welt der Programmierung begeistert und haben von Frau Dr. Petram das Angebot erhalten, bei ihr das Bachelorprojekt durchzuführen. Da wir im Laufe unserers Studiums viele Kurse bei ihr besucht haben, unter anderem ``KI - maschinelles Lernen'' und ``Big Data'' waren wir sofort begeistert, als wir gehört haben, mit dem humanoiden Roboter Pepper arbeiten zu dürfen.\\

Wir haben von ehemaligen Masterstudierenden eine Vorführung einer laufenden Anwendung bekommen und haben uns natürlich direkt gefragt, was wir denn tolles entwickeln können, damit nicht nur wir, sondern vor allem auch die Hochschule einen Mehrwert davon erhält.

Aufgrund dieser Fragestellung haben wir uns dazu entschieden, dass wir zwei Anwendungsfälle zum Einsatz des Peppers implementieren wollen, um diese auch mit Hinblick auf den Tag der Informatik vorstellen zu können. Diese Anwendungsfälle sind zum einen die App für Pepper, welche der Informationsgewinnung im Bezug auf die Hochschule und das Studentenleben dient, zum anderen wollen wir mit dieser App Daten über die geführten Gespräche sammeln, um das Potential eines solchen Roboters aufzuzeigen.\\