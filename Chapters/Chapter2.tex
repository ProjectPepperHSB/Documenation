\chapter{Motivation und Zielsetzung}
\label{sec:Motivation-und-Zielsetzung}
\lhead{Kapitel 2. \emph{Motivation und Zielsetzung}}

Wie alle Bachelorstudierende kommt irgendwann die Zeit, in der es darum geht, ein Projekt
zu finden, mit welchem man sich über einen längeren Zeitraum beschäftigt, um dies im Rahmen
des Studiums schriftlich niederzulegen. Wir drei sind von der Welt der Programmierung begeistert
und haben von Frau Dr. Petram das Angebot erhalten, bei ihr das Bachelorprojekt durchzuführen.
Da wir im Laufe unserers Studiums viele Kurse bei ihr besucht haben, unter anderem
\grqq{}KI - maschinelles Lernen\grqq{} und \grqq{}Big Data\grqq{} waren wir sofort begeistert, als wir
gehört haben, mit dem humanoiden Roboter Pepper arbeiten zu dürfen.\\

Wir haben von ehemaligen Masterstudierenden eine Vorführung einer laufenden Anwendung bekommen und haben
uns natürlich direkt gefragt, was wir denn tolles entwickeln können, damit nicht nur wir, sondern auch die
Hochschule den meisten Mehrwert davon erhalten.

Aufgrund dieser Fragestellung haben wir uns dazu entschieden, dass wir zwei Anwendungsfälle zum Einsatz des
Peppers implementieren wollen, um diese auch mit Hinblick auf den Tag der Informatik vorstellen zu können.\\

Neben der Entwicklung dieser Anwendungsfälle wollen ebenfalls unser Verständnis für die Thematik KI und
Big Data ausbauen, indem wir ein Backend schaffen, welches mit Hilfe von Pepper Daten sammelt, um diese in
einem späteren Prozess anderweitig zu verwenden.