\chapter{Abschließende Worte}
\label{sec:zusammenfassung}
\lhead{Kapitel 10. \emph{Abschließende Worte}}

Die Arbeit mit Pepper hat uns die Relevanz und den großen Nutzen eines humanoiden Roboters 
verdeutlicht. Außerdem hat sich für uns gezeigt, dass vor allem die Interaktion zwischen dem 
Benutzer und der Software durch einen humanoiden Roboter stark verbessert wird. 

Wir haben im Laufe des Projektes sehr viele und zum Teil neue Erfahrungen mit verschiedenen Softwares und Tools sammeln können. Die Entwicklung einer Anwendung für einen auf Andriod basierenden Roboter, sowie die vollständige Entwicklung von zwei Webservern mit unterschiedlichen Backends, haben uns aufgezeigt, welches Potential uns durch die freie Gestaltung unseres Bachelorprojektes geboten wurde. Dies gilt auch für die kleinen Skripte, die wir zur Ermittlung der Mensa- oder Routeninformationen entwickelt haben.

Auch die Arbeit im Team war sehr angenehm. Wir haben zum Teil täglich miteinander
den Diskurs gesucht, neue Ideen ausgetauscht miteinander gearbeitet. Hierbei konnte
jeder etwas von dem Anderen lernen, sodass nach einem Treffen alle etwas schlauer
und mit einem guten Gefühl den Jitsi Raum verlassen konnten.

Das Potential von Pepper ist größer als wir zu Anfang dachten, da dieser für die verschiedensten Anwendungsfälle 
programmiert werden kann. Bei der Planung der einzelnen Anwendungsfälle für die Hoschule Bremerhaven ist uns
dies schon aufgefallen, da wir viele weitere Ideen bekommen haben, was noch implementiert werden könne. Viele
Dieser Ideen würden wären jedoch über den Rahmen eines Bachelorprojektes hinaus gegangen. Hätten wir die Entwicklungsumgebung
ROS verwenden können, wäre das Potential noch größer, da man auch auf die einzelnen Kameras und Sensonren zuzugreifen.