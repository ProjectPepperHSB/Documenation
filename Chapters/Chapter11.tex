\chapter{Features}
\label{sec:MyGain-Features}
\lhead{Kapitel 11. \emph{MyGain - Features}}

\section{Allgemein}
Neben den bisher besprchenen Seiten haben wir zwei weitere erstellt, welche wir jedoch nur am Rand erwähnen wollen, da diese eher dazu gedacht sind, um das Softwareprojekt etwas realistischer aussehen zu lassen, als dass sie einen wirklichen Mehrwert im Bezug auf die Interaktionsmöglichkeiten bieten.\\

\section{Einstellungen}
Es gibt sehr viele Arten von Menschen - manche haben ein Passwort für alles, manche ändern ihre Passwörter einmal im Monat und manche mehrmals am Tag. Damit wir auch solchen Kunden gerecht werden können, haben wir die Seite \code{settings.php} erstellt, welche es angemeldeten Kunden ermöglicht, ihr Passwort zu ändern. \\

\begin{figure}[H]
    \begin{center}
        \includegraphics[scale=0.5]{settingsseite}
        \caption{Settings - Seite}
        \label{fig:settingssite}
    \end{center}
    \centering
\end{figure}

Sie ist in der Menüleiste zu finden und zeigt dem Nutzer seine bei uns hinterlegten Daten. Dazu gehören der Benutzername und die E-Mail Adresse, welche zur Registration verwendet wurde.

Besteht der Wunsch zur Änderung des Passwortes, so kann dies hier durchgeführt werden, indem man die üblichen Eingabefelder zur Änderung des Passwortes ausfüllt und bestätigt.

Während des Softwareprojektes haben wir auch probiert, bei Registration eine E-Mail an die hinterlegte Adresse zur Verifikation zu senden, jedoch ist die Ausführung der php Methode \code{mail}, in der Infrastruktur der Hochschule Bremerhaven aus Sicherheitsgründen deaktivert.\\

\begin{figure}[H]
    \begin{center}
        \includegraphics[width=\textwidth]{sendMail}
        \caption{Wie man eine Mail mit PHPMailer sendet (sendMail.php}
        \label{fig:sendMail}
    \end{center}
    \centering
\end{figure}

Auf Nachfrage bei Prof. Dr. Radfelder wurde uns mitgeteilt, dass es alternative Bibliotheken wie \href{https://github.com/PHPMailer/PHPMailer}{PHPMailer} gibt, die es uns erlauben über einen separaten E-Mail Anbieter Nachrichten zu versenden. Da viele E-Mail Anbieter ihre Dienste nur für private Zwecke anbieten und wir kein Risiko eingehen wollten, haben wir Interaktionen, die den E-Mail Verkehr voraussetzen, nicht implementiert.

Da dies jedoch ein spannendes Thema ist, haben wir die Bibiliothek dennoch eingebunden, nutzen den Service aber nicht. In Abbildung \ref{fig:sendMail} ist zu sehen, wie wir Kunden eine Mail schicken würden, wenn sie Kontakt mit unserem Support Team aufgenommen hätten.

Der Nutzen einer solchen Implementierung liegt auf der Hand. Mit der Möglichkeit des Kontakts per E-Mail wäre eine Implementierung von Techniken zum Zurücksetzen des Passwortes oder auch die Verifikation ein klarer Vorteil gegenüber der jetzigen Situation. Auch die Bestätigung einer Kontaktanfrage, sowie des Erhalts einer Antwort auf jene wäre somit gewährleistet. \\

\section{News}
Wie jede Website, die was von sich hält, haben auch wir einen Bereich eingerichtet, auf welchem wir Nachrichten und Neuigkeiten mitteilen. Dies findet auf der Seite \code{news.html} statt. Auf dieser werden Ankündigungen mitgeteilt, sodass jeder sie begutachten kann.
