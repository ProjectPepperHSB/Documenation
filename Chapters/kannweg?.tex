% \chapter{Einrichtung des Laptops}
% \label{sec:laptop}
% \lhead{Kapitel 7. \emph{Einrichtung des Laptops}}

% \section{Technische Details} % Als Tabelle wäre das schöner
% \begin{itemize}
%     \item Intel Core i9-11900K Prozessor (bis zu 5,3 GHz), Octa-Core
%     \item 43,9 cm (17,3") Full HD 16:9 Display (entspiegelt), Webcam
%     \item 32 GB RAM, 1.500 GB SSD, Fingerprint
%     \item 1,5TB SSD
%     \item NVIDIA GeForce RTX3080 Grafik (16384 MB), HDMI, Thunderbolt 4, MiniDP
%     \item Windows 10 Professional 64 Bit, 4,6 kg
% \end{itemize}


% Link:
% \url{https://www.cyberport.de/?DEEP=1C10-3HP&APID=276&gclid=CjwKCAjwhaaKBhBcEiwA8acsHBI1eAIk502Db7YTMnAbge8DG6400wSbpdoIzbsnjSahPU_gHRYDqBoC2FoQAvD_BwE}

% \section{Allgemein}
% Zu Beginn des Projekts haben wir einen XMG Ultra 17,3 Laptop erhalten, den wir beliebig verwenden durften. Wir haben uns Gedanken darüber gemacht, wie wir den Laptop am Sinnvollsten verwenden können und welches Betriebssystem wir verwenden. 

% Wir haben uns dazu entschieden, den Laptop als Speicherort für die Daten zu benutzten, die wir mit Pepper generieren und speichern wollen. Außerdem soll der Laptop als gemeinsames System zur Kommunikation und Datentransfer dienen. Dafür war die Überlegung, ein Nextcloud Server zu implementieren und Open SSH zu verwenden um auch von außerhalb Zugriff zu gewährleisten. Des Weiteren hatten wir auch die Idee, den Laptop als Workstation zu benutzten, um gemeinsam in Android Studio zu arbeiten oder weitere Applikationen wie Python Anaconda zu nutzen. 

% Die Wahl des Betriebssystems fiel uns schwer, da jeder seine eigenen Vorlieben besaß. Am Ende konnten wir uns allerdings eindeutig entscheiden. 
% Für den Nextcloud Server, sowie das Speichern der Daten in eine Datenbank haben wir Ubuntu Server verwendet. Hier war auch die Überlegung die Desktop version zu verwenden, allerding bräuchten wir diesen für unser Vorhaben nicht unbedingt und zur Not hätte man sich den Desktop später immer noch einrichten können. Der Vorteil an Ubuntu Server ist, dass man von Null anfängt und keine unnötigen Applikationen bereits Installiert sind, so wie es mit der Ubuntu Desktop Version der Fall ist. 

% Da wir Ubuntu Server nicht als Betriebssystem für die Workstation verwenden konnten haben wir uns entschieden eine zweite Partition zu erstellen und Windows 10 einzurichten.

% \textbf{Einrichtung der Betriebssysteme}
% ...

% muss das wirklich sein? - das hier ist doch keine Bedienungsanleitung