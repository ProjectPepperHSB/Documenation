\chapter{Das Backend}
\label{sec:das-backend}
\lhead{Kapitel 6. \emph{Das Bbackend}}

\section{Allgemein}
In Kapitel \textbf{KAPITEL VERWEIS EINFÜGEN} und \textbf{KAPITEL VERWEIS EINFÜGEN} haben wir das Thema Backend Services
des öfteren erwähnt, daher werden wir uns nun diesem Bereich widmen.

Wir als Studenten haben durch das Studium, aber auch durch Nebenjobs und private Projekte Erfahrungen
mit verschiedenen Programmierkonzepten sammeln können. Da wir bei der Entwicklung der Pepper Anwendung
jedoch nur zwischen den Sprachen Kotlin und Java entscheiden konnten, haben wir zu Beginn unseres Projektes,
das Gefühl gehabt, in unserer Entwicklung, was die Nutzung und Implementierung verschiedener
Methoden angeht sehr eingeschränkt zu sein.

Genau hier war es uns von Vorteil, Kenntnisse zu NodeJS und Javascript gesammelt zu haben, denn damit
ist es uns möglich gewesen, einen Express Server mit Hilfe der Laufzeitumgebung NodeJS aufzusetzen,
auf welchem wir vielfältige Möglichkeiten zur Implementierung von Komplexen Vorgängen haben.

\section{Implementierung}
Damit wir eine Anwendung erstellen können, auf die wir über eine URL von außen zugreifen können,
brauchen wir natürlich einen Webserver. Da wir in der Hochschule verschiedene Server benutzen dürfen,
haben wir uns mit Prof. Dr. Oliver Radfelder abgesprochen, welcher uns daraufhin einen
gemeinsamen Benutzeraccount mit Namen \grqq{}hbv-kms\grqq{} auf dem Server Hopper
angelegt hat.
Für unseren gemeinsamen Account steht ein Dockercontainer zur Verfügung, auf welchem wir unsere
Anwendung bereitstellen. Von außen ist unsere Anwendung dann über die URL
\href{https://informatik.hs-bremerhaven.de/docker-hbv-kms-http}{https://informatik.hs-bremerhaven.de/docker-hbv-kms-http}
erreichbar.

\subsection{NodeJS, npm, Express}
Unsere Webanwendung basiert auf NodeJS und JavaScript. NodeJS ist eine Laufzeitumgebung, welche JavaScript außerhalb des Browsers
ausführen kann und somit wichtige Prozesse auf dem Server anstatt beim Client ausführen kann. NodeJS hat einen eigenen Paketmanager,
npm (Node Package Manager), mit welchem sich vielfältige Libraries und Frameworks installieren und ausführen lassen.

... wir nutzen das Modul Express, welches es uns ermöglicht, ohne großen Aufwand, eine vollig anpassbare Webanwendung zu erstellen.



\subsection{Endpunkte}
Endpunkte, auch Routes genannt, sind Schnittstellen, an denen man von außen zugreifen kann.
Ein Beispiel für eine Route innerhalb der Anwendung ist: /docker-hbv-kms-http/timetable.
Anzumerken ist, dass jede unserer Routes mit /docker-hbv-kms-http beginnt, da dies das
der Stamm unseres Anwendungspfades innerhalb des Hochschulwebservices ist.

\section{Installation}
erst dies, dann dass und ananas
ggf. muss das Ddeploy Skript angepasst werden
