\chapter{Erste Schritte: Das Grundgerüst}
\label{sec:erste-schritte-und-installation}
\lhead{Kapitel 3. \emph{Erste Schritte: Das Grundgerüst}}

\section{Allgemein}
Wir haben bisher noch kaum Berührung mit Robotern im Alltag erlebt und daher ist unsere Erfahrung im Bereicht
der Roboterprogrammierung sehr begrenzt. Zum Glück haben uns ehemalige Masterstudierende einen schnellen
Einstieg ermöglicht, denn diese habe eine Anwendung für das Unternehmen \grqq{}Erlebnis Bremerhaven\grqq{}
entwickelt und uns deren Quellcode zur Verfügung gestellt.

Wir haben sehr schnell gemerkt, dass uns dies zwar eine gute Hilfe für den Anfang ist, jedoch gibt es mittlerweile
deutlich effizientere Methoden einzelne Komponenten miteinander zu verknüpfen.

\section{Wer ist dieser Pepper?}
%
% wer ist pepper, was kann er, von wem entwickelt, wie teuer, wo zu kaufen, wo wenden ihn andere an
% welche vor und nachteile bietet er
%
%
%
% \begin{quote}
% ``\cite{brokerdefinition} Ein Broker ist eine Person oder Firma, die als Vermittler zwischen einem Anleger und einer Wertpapierbörse agiert. Da Wertpapierbörsen nur Aufträge von Einzelpersonen oder Firmen annehmen, die Mitglieder dieser Börse sind, benötigen einzelne Händler und Anleger die Dienste von Börsenmitgliedern. Makler erbringen diese Dienstleistung und werden auf verschiedene Weise entlohnt, entweder durch Provisionen, Gebühren oder durch die Bezahlung durch die Börse selbst." 
% \end{quote}

\section{Projektmanagement}
Um dieses Projekt erfolgreich durchzuführen und zielorientiert zur arbeiten, treffen wir uns wöchentlich
in der Hochschule und Besprechen unsere neuesten Ideen und Implementierungen. Auch zwischendurch
stehen wir im Kontakt, um Probleme und Schwierigkeiten schnell zu beheben.

Damit wir gemeinsam an einer Codebasis arbeiten können, haben wir zu Anfang unseres Projektes
die Organisation \href{https://github.com/ProjectPepperHSB}{HBV-Pepper} auf
\href{https://github.com}{GitHub} angelegt. Jegliche Beteiligung, sowie verschiedene Versionen können dort eingesehen werden.
Zum Ende des Projektes haben sich hier 4 Repositories von uns angesammelt, welche folgende Themen beinhalten:
\begin{itemize}
    \item Pepper Anwendung
    \item Backend Serveranwendung
    \item Dokumentation
    \item Sonstiges
\end{itemize}

Das erste Repo beinhaltet unsere lauffähige Anwendung, welche die von uns
implementierten Anwendungsfälle, sowie sonstige Funktionalitäten für den Roboter Pepper beinhaltet.

Der Backend Serveranwendung auf welchem wir in Kapitel <HIER LINK SETZEN> eingehen, ist unsere Schnittstelle, mit welcher wir
Informationen, die Pepper durch seine Interaktionen mit der Umgebung sammelt, abfangen und speichern. Zudem Versorgen wir
Pepper mit Informationen, Beispiel hierfür sind Informationen zum Stunden- und Mensaplan.

Das Repositorie zur Dokumenation beinhaltet alle zu diesem Dokument gehörenden Dateien. Da wir diese mit LaTeX anfertigen,
lohnt es sich auch dies zu versionieren.

Für kleine Skripte, Tests und andere zu keinem festen Thema dazugehörigen Dokumente haben wir das Repositorie
\grqq{}Sonstiges\grqq{} angelegt.

Wir werden in diesem Dokument neben der Implementierung auch auf das Herunterladen, Installieren und Einrichten
unserer Software eingehen, damit Studierende, die dieses Projekt weiterführen wollen, die Möglichkeit
haben, mit einem etwas fortgeschrittenem Projekt zu starten. Darüber hinaus befindet sich in jedem Stammverzeichnis
eine Readme, welche Anforderungen und erste Schritte aufzeigen, um einen schnellen Einstieg zu ermöglichen.
