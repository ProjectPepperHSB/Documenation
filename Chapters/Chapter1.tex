\newcommand{\chapterdoc}{Kapitel 1. }
\chapter{Dokumentation und Struktur}
\label{Dokumentation und Struktur}
\lhead{\chapterdoc \emph{Dokumentation und Struktur}}

Diese Dokumentation ist in XXX Kapitel gegliedert, welche sich mit den einzelnen Aspekten unseres Projektes auseinandersetzen.
Die Stuktur sieht wie folgt aus:

\begin{itemize}
	\item \textit{Kapitel 1}: Dokumentation und Struktur
	\item \textit{Kapitel 2}: Motivation und Zielsetzung
	\item \textit{Kapitel 3}: Erste Schritte und Installation
	\item \textit{Kapitel 4}: Anwendungsfall -  Hochschule
	\item \textit{Kapitel 5}: Anwendungsfall - XXX
	\item \textit{Kapitel 6}: Das Backend
	\item \textit{Kapitel 7}: Möglichkeiten der Erweiterung % daten des backends nutzen, pepper - dessen ki (bild und sprache) aufzeigen
	      % \item \textit{Kapitel 8}: Features % <-- was haben wir noch ausgelassen?
	\item \textit{Kapitel 12}: Abschließende Wort
\end{itemize}

Zu sehen ist, dass wir nach der Erläuterung unserer Struktur und des Grundes für diese Arbeit
von vorn bis hinten durch unser Projekt gehen. Nachdem wir unser Grundgerüst
und unsere ersten Schritte einschließlich Installation dargelegt haben, werden wir zwei Anwendungsfälle zum Einsatz von Pepper
Aufzeigen, sowie die von uns Implementierten Funktionen aufzeigen.

Ebenfalls werden wir die Herangehensweise, sowie die technische Umsetzung und Hürden, mit denen wir zu kämpfen gehabt haben erläutern, um alle Aspekte
der Implementierung von Software für den Roboter Pepper abzudecken.

Nachfolgend haben wir die Fragestellungen festgehalten, auf welche wir in den jeweiligen Abschnitten eingehen werden.

\begin{table}[H]
	\centering
	\begin{tabular}{ll}
		% \toprule
		Kapitel                            & Fragestellung                                    \\
		\midrule
		1. Dokumentation und Strktur       & Wie ist diese Arbeit aufgebaut?                  \\
		2. Motivation und Zielsetzung      & Was wollen wir hiermit erreichen?                \\
		3. Erste Schritte und Installation & Was ist das Fundament unseres Softwareprojektes? \\
		4. Anwendungsfall - Hochschule     & Wie kann Pepper der Hochschule helfen?           \\
		6. Node - Express Webanwendung     & Welche Prozesse laufen im Hintergrund?           \\
		6. Skripte und Erweiterungen       &                                                  \\
		7. Skripte und Erweiterungen       &                                                  \\
		% 11. Features & Was haben wir sonst nocht nicht erwähnt?\\
		12. Abschließende Worte            & Was haben wir aus diesem Projekt mitgenommen?    \\
		\bottomrule
	\end{tabular}
	%\caption{Flow in the logic}
	\label{tab:logic_flow}
\end{table}

Im Anschluss dieser Dokumenation ist das Literaturverzeichnis mit allen Referenzen zu finden.


\textbf{ES GIBT REPOSITORIES, IN DENEN ALLES EINSEHBAR IST, WENN DIE NICHT MEHR ZUR VERFÜGUNG STEHEN, HAT DIE
	HOCHSCHULE DEN GESAMTEN KRAM.
	DER QUELLCODE IST WEITESGEHEND DOKUMENTIERT
	WIR VERZICHTEN HIER GRÖßTENTEILS AUF DIE DARSTELLUNG DES QUELLCODES, DA ES ZU VIEL WÄRE.
	AN MANCHEN STELLEN ZEIGEN WIR GEWISSE DINGE AUF, DIES IST ABER NUR EIN KLEINER TEIL UND KANN
	IN DEN ABGEGEBENNEN DATEIN UND ODER AUF GITHUB EINGESEHEN WERDEN
}