\newcommand{\chapterdoc}{Kapitel 1. }
\chapter{Dokumentation und Struktur}
\label{Dokumentation und Struktur}
\lhead{\chapterdoc \emph{Dokumentation und Struktur}}

Diese Dokumentation ist in 11 Kapitel gegliedert, welche sich mit den einzelnen Aspekten unseres Projektes auseinandersetzen.
Die Stuktur sieht wie folgt aus:

\begin{itemize}
    \item \textit{Kapitel 1}: Dokumentation und Struktur
    \item \textit{Kapitel 2}: Motivation und Zielsetzung
    \item \textit{Kapitel 3}: Erste Schritte: Das Grundgerüst
    \item \textit{Kapitel 4}: Konzept: Pepper App für die Hochschule
    \item \textit{Kapitel 5}: Implementierung: Pepper App
    \item \textit{Kapitel 6}: Implementierung: Anwendungsfälle
    \item \textit{Kapitel 7}: Node - Express Webanwendung
    \item \textit{Kapitel 8}: Skripte und Erweiterungen
    \item \textit{Kapitel 9}: Big Data
    \item \textit{Kapitel 10}: Ausblick
    \item \textit{Kapitel 11}: Abschließende Worte
\end{itemize}

Zu sehen ist, dass wir nach der Erläuterung unserer Struktur und des Grundes für diese Arbeit von vorn bis hinten durch unser Projekt gehen. Nachdem wir unser Grundgerüst und unsere ersten Schritte einschließlich Installation dargelegt haben, werden wir zwei Anwendungsfälle zum Einsatz von Pepper
aufzeigen, sowie die von uns Implementierten Funktionen darlegen.

Ebenfalls werden wir die Herangehensweise, sowie die technische Umsetzung und Hürden, mit denen wir konfrontiert worden zeigen, um alle Aspekte
der Implementierung von Software für den Roboter Pepper abzudecken.

Nachfolgend haben wir die Fragestellungen festgehalten, auf welche wir in den jeweiligen Abschnitten eingehen werden.

\begin{table}[H]
    \centering
    \resizebox{\textwidth}{!}{%
        \begin{tabular}{ll}
            % \toprule
            Kapitel                                   & Fragestellung                                             \\
            \midrule
            1. Dokumentation und Strktur              & Wie ist diese Arbeit aufgebaut?                           \\
            2. Motivation und Zielsetzung             & Was wollen wir hiermit erreichen?                         \\
            3. Erste Schritte: Das Grundgerüst        & Was ist das Fundament unseres Softwareprojektes?          \\
            4. Konzept: Pepper App für die Hochschule & Wie kann Pepper der Hochschule helfen?                    \\
            5. Implementierung: Pepper App            & Wie wurde unsere Applikation in Android Studio umgesetzt? \\
            6. Implementierung: Anwendungsfälle       & Wie wurden unsere Anwendungsfälle realisiert?             \\
            7. Node - Express Webanwendung            & Welche Prozesse laufen im Hintergrund?                    \\
            8. Skripte und Erweiterungen              & Welche Programme haben wir zusätzlich geschrieben?        \\
            9. Big Data                               & Was sagen uns die gesammelten Daten?                      \\
            10. Ausblick                              & Wie könnte es weiter gehen?                               \\
            11. Abschließende Worte                   & Was haben wir aus diesem Projekt mitgenommen?             \\
            \bottomrule
        \end{tabular}}
    \caption{Kapitel und Fragestellung}
    \label{tab:logic_flow}
\end{table}

Unser Projekt ist in Github versioniert. Im Anschluss befindet sich eine Auflistung der einzelnen Github Repositories:

\begin{table}[H]
    \centering
    \resizebox{\textwidth}{!}{%
        \begin{tabular}{ll}
            % \toprule
            Repository                & erreichbar unter                                               \\
            \midrule
            Pepper Applikation        & \url{https://github.com/ProjectPepperHSB/hbv-pepper-app}       \\
            Webanwendung              & \url{https://github.com/ProjectPepperHSB/NodeJS_Server4Pepper} \\
            Realtime Dashboard        & \url{https://github.com/ProjectPepperHSB/WebsocketServer}      \\
            Scripte und Erweiterungen & \url{https://github.com/ProjectPepperHSB/Backend-Services}     \\
            Dokumentation             & \url{https://github.com/ProjectPepperHSB/Documenation}         \\
            \bottomrule
        \end{tabular}}
    \caption{Github Repositories}
    \label{tab:githublinks}
\end{table}

Sollten die Inhalte über die Links nicht mehr erreichbar sein, kann sich an die Autoren oder an die Hochschule Bremerhaven gewendet werden.

Im Anschluss dieser Dokumenation ist das Literaturverzeichnis mit allen Referenzen zu finden.